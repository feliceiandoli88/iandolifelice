 \documentclass[margin,line,pifont,palatino,courier]{res}
%\usepackage{amsfont}
\usepackage{pifont}
\usepackage[latin1] { inputenc}
\usepackage{amssymb}

%\usepackage{amsfont}
%\topmargin .5in
%\oddsidemargin -.5in
%\evensidemargin -.5in
%\textwidth=6.0in
 \textheight=9.0in
%\itemsep=0in
%\parsep=0in
\usepackage{fancyhdr}
%\topmargin=0in
%\textheight=8.5in
\pagestyle{fancy}
\renewcommand{\headrulewidth}{0pt}
\fancyhf{}
%\cfoot{\thepage}
%\lfoot{\textit{\footnotesize Research Statement}}
\rfoot{{\footnotesize Curriculum Vitae, Felice Iandoli, \thepage}}


\newenvironment{list1}{
  \begin{list}{\ding{113}}{%
      \setlength{\itemsep}{0in}
      \setlength{\parsep}{0in} \setlength{\parskip}{0in}
      \setlength{\topsep}{0in} \setlength{\partopsep}{0in}
      \setlength{\leftmargin}{0.17in}}}{\end{list}}
\newenvironment{list2}{
  \begin{list}{$\bullet$}{%
      \setlength{\itemsep}{0in}
      \setlength{\parsep}{0in} \setlength{\parskip}{0in}
      \setlength{\topsep}{0in} \setlength{\partopsep}{0in}
      \setlength{\leftmargin}{0.2in}}}{\end{list}}

\begin{document}

\name{Felice Iandoli \vspace*{.1in}}

\begin{resume}

\section{\sc Contact Information}

\vspace{.05in}
\begin{tabular}{@{}p{2.75in}p{2in}}
Laboratoire Jacques Louis Lions & +33 752506938 \\
  4 Place Jussieu, 75005 Paris, France                  & \verb+iandoli@ljll.math.upmc.fr+\\
                       \end{tabular}

\section{\sc Research Interests}
Microlocal Analysis, Dispersive Estimates, Wave and Schr\"odinger Equations, Dynamical Systems, Nonlinear Dispersive PDEs, Normal Forms

\section{\sc Career and Education}

{\bf Laboratoire J.L. Lions, Sorbonne Universit\'e}\\
\vspace*{-.2in}
\begin{list1}
\item[] Post-doctoral researcher funded by {ERC ANADEL}, \emph{starting from October 1-st, 2019}.
\end{list1}
{\bf Laboratoire J.A. Dieudonn\'e, Universit\'e de Nice }\\
\vspace*{-.2in}
\begin{list1}
\item[] Post-doctoral researcher funded by {ERC ANADEL},  \emph{November 1-st, 2018}-\emph{30 September, 2019}.
\end{list1}
{\bf SISSA}\\
\vspace*{-.2in}
\begin{list1}
\item[] Ph.D. in Mathematical analysis, models and applications.\\
The thesis has been defended on 26/09/2018, the examination has been passed \emph{cum laude}. 
\begin{list2}
%\vspace*{-.2in}
\item Title of the thesis:  Local and almost global solutions for  fully-nonlinear Schr\"odinger equations on the circle
\item Advisors:  Prof. Massimiliano Berti and Dr. Roberto Feola
\end{list2}
%\item[] M.S.~in Mathematics, May 1996
\end{list1}

{\bf University of Pisa}\\
\vspace*{-.2in}
\begin{list1}
\item[] Master degree in mathematics, grade: 110/110 \emph{cum laude}

\begin{list2}
%\vspace*{.05in}
\item Title of dissertation: Teoria di scattering per NLS (eng: Scattering theory for NLS)
\item Advisor: Prof. Nicola Visciglia
\end{list2}

\end{list1}






\section{\sc Publications}

$1)${\rm{F. IANDOLI, R. Scandone}},  \textit{Dispersive estimates for Schr\"odinger operators with point interactions in $\mathbb{R}^3$,}  {\textbf {Advances in Quantum Mechanics: Contemporary Trends and Open Problems}}, A. Michelangeli and G. Dell'Antonio, eds., Springer INdAM Series, vol. 18, Springer International Publishing, (2017).\vspace{0.1cm}\\
$2)${\rm{R. Feola, F. IANDOLI}},  \textit{Local well-posedness for quasi-linear NLS with large Cauchy data on the circle,}  \textbf{Annales de l'Institut Henri Poincar\'e (C) Non Linear Analysis}, (1) Vol 36: 119-164, (2019).\vspace{0.1cm}\\
$3)${\rm{R. Feola, F. IANDOLI}}, \textit{Long time existence for fully nonlinear NLS with small Cauchy data on the circle},   \textbf{Ann. Sc Norm.  Pisa Cl. Sci.} (5), Vol XXII, 109-182, (2021).\vspace{0.1cm}\\
$4)${\rm{J., Bernier, R. Feola, B. Gr\'ebert, F.IANDOLI}}, \textit{Long-time existence for semi-linear beam equations on irrational tori},  \textbf{J. Dyn. Diff. Equat.} (3), Vol 33, 1363-1398, (2021). \vspace{0.1cm}\\
$5)${\rm{R. Feola, F. IANDOLI}}, \textit{A non-linear Egorov theorem and Poincar\'e-Birkhoff normal forms for quasi-linear pdes on the circle}, preprint: arxiv.org/abs/2002.12448, (2020).\vspace{0.1cm}\\
$6)${\rm{R. Feola, F. IANDOLI}}, \textit{Local well-posedness for the quasi-linear Hamiltonian Schr\"odinger equation on tori}, preprint: arxiv.org/abs/2003.04815, (2020). (\textbf{Accepted on Journal de Math\'ematiques pures et appliqu\'ees})\vspace{0.1cm}\\
$7)${\rm{R. Feola, B. Gr\'ebert, F.IANDOLI}}, \textit{Long time solutions for quasi-linear Hamiltonian perturbations of Schr\"odinger and Klein-Gordon equations on tori}, (\textbf{Accepted on Analysis and PDES}),preprint: arxiv.org/\\abs/2009.07553, (2020). \vspace{0.1cm}\\
$8)${\rm{R. Feola, F. IANDOLI, F. Murgante}}, \textit{Long-time stability of the quantum hydrodynamical system on irrational tori}, \textbf{Math. in Engineering}, (3), Vol 4, 1-24, (2022).\vspace{0.1cm}\\



\section{\sc Invited Conferences}

Invited speaker at:\\
 $\bullet$ \emph{Dynamics of nonlinear dispersive PDE's}, February 2018, La Thuile, Italy, Invited by Prof. Nicola Visciglia.\\
$\bullet$  \emph{Nonlinear Dispersive PDE's}, October 2018, Universit\`a Sapienza, Rome, Italy, Invited by Prof. Oana Ivanovici.\\
  $\bullet$ \emph{Hamiltonian PDEs and nonlinear waves}, February 2019, La Thuile, Italy, Invited by Dr. David Lafontaine.\\



\section{\sc Seminars}
$\bullet$ \emph{Existence en temps grand pour l'\'equation de Klein-Gordon sur les tores}, 2021, S\'eminaire \`a l'Universit\'e de Cergy Paris.\\
$\bullet$ \emph{Existence en temps grand pour l'\'equation de Klein-Gordon sur les tores}, 2022, S\'eminaire \`a l'Universit\'e de Besan\c{c}on.\\
%$\bullet$ \emph{Existence en temps grand pour l’\'equation de Klein-Gordon sur les tores}, 2021, S\'eminaire \'a l'universit\'e de Besançon. 
%$\bullet$ \emph{Existence en temps grand pour l’\'equation de Klein-Gordon sur les tores}, 2021, S\'eminaire CY Cergy Paris Universit\'e.
$\bullet$ \emph{Local and almost global solutions for quasi-linear Schr\"odinger equations}, 2020, S\'eminaire Enriques-Lebesgue, Milano-Nantes, via ZOOM. \\
$\bullet$ \emph{Long time existence for small solutions of Hamiltonian or reversible quasilinear equations on the circle}, 2020, S\'eminaire de l'\'equipe EDP, IECL, Nancy, France.\\
$\bullet$ \emph{Long time solutions for the fully-nonlinear NLS on the circle}, 2020, S\'eminaire du LAGA,  Paris 13, Paris, France.\\
$\bullet$ \emph{Local and almost global solutions for fully non-linear Schr\"odinger equations on the circle}, 2018, Laboratoire  J.A. Dieudonn\'e, Nice, France.\\
$\bullet$ \emph{On the quasi-linear Schr\"odinger equations on the circle}, 2018, Universit\`a di Pisa, Pisa, Italy.\\

\section{\sc Visiting researcher}

From October 1-st, to 31 October, 2018, Laboratoire  J.A. Dieudonn\'e, Nice, France,  Invited by Prof. Oana Ivanovici.

\section{\sc Attended conferences \\}
 $\bullet$ \emph{Normal forms and large time behavior for nonlinear PDE}, 2015, IHES, Bures-sur-Yvette, France.  \\
 $\bullet$  \emph{Nonlinear Waves 2016: Summer School}, 2016, Centre Henri Lebesgue, Nantes, France.  \\
$\bullet$ \emph{Hamiltonian Dynamics, PDE's and Waves on the Amalfi coast
}, 2016, Maiori, Italy. \\
$\bullet$ \emph{Winter School ``Dynamics and PDE's"}, 2017, Saint-Etienne de Tin\'ee, France. \\
$\bullet$ \emph{Linear and Nonlinear Wave Phenomena: Stability, Propagation of Regularity and Turbulence}, 2018, Cortona, Italy.\\
$\bullet$ \emph{Quantum Resonances and Related Topics (conference in honor of Andr\'e Martinez)}, 2019, Paris, France. \\
$\bullet$ \emph{Dispersive Waves and Related Topics (conference in honor of Gilles Lebeau)}, 2019, Bergen, Norway.\\
$\bullet$ \emph{New Trends in Propagation of Linear and Nonlinear Wave Phenomena}, 2019, Erice, Italy. \\
$\bullet$ \emph{Journ\'ees \'equations aux d\'eriv\'ees partielles}, 2021, Obernai, France.\\
$\bullet$ \emph{Qualitative properties of dispersive PDEs}, 2021, Rome, Italy.\\

\section{\sc Experience as a peer-reviewer} 
$\bullet$ Discrete and continuous dynamical systems\\
$\bullet$ Mathematics in Engineering\\ 
$\bullet$ MDPI Mathematics\\

\section{\sc Teaching (in french)} $\bullet$ 36h TD: \emph{S\'eries et S\'eries de fonctions}, 2020-2021, UPMC (Sorbonne Universit\'e), L2\\
$\bullet$ 36h TD: \emph{\'Equations diff\'erentielles}, 2020-2021, UPMC (Sorbonne Universit\'e), L2\\
$\bullet$ 36h TD: \emph{Topologie et calcul diff\'erentiel}, 2021-2022, UPMC (Sorbonne Universit\'e), L2\\


\section{\sc Spoken languages} \begin{list2}
\vspace*{.05in}
\item  Italian: mother tongue
\item English: fluent
\item French: Utilisateur exp\'eriment\'e
\end{list2}




\end{resume}
\end{document}
